% \chapter{How to write a research paper}
% The final deliverable of a robotics class often is a write-up on a ``research'' project, modeled after research done in industry or academia. Roughly, there are three classes of papers:

\chapter{如何写研究论文}
机器人类的最终交付通常是在“研究”项目上的一个写作,在工业或学术界进行研究之后建模。 大概有三类论文:

\begin{enumerate}
% \item Original research
% \item Tutorial
% \item Survey

\item 原创研究
\item 教程
\item 综述
\end{enumerate}

% The goal of this chapter is to provide guidelines on how to think about your project as a research project and how to report on your results as original research.

本章的目标是提供关于如何将您的项目作为研究项目进行思考的指导原则,以及如何将您的结果报告为原始研究。

% \section{Original}
% Classically, a scientific paper follows the following organization:

\section{原创研究}
经典地,科学论文遵循以下组织:

\begin{enumerate}
% \item Abstract
% \item Introduction
% \item Materials \& Methods
% \item Results
% \item Discussion
% \item Conclusion

\item 摘要
\item 前言
\item 材料 \& 方法
\item 结果
\item 讨论
\item 总结
\end{enumerate}

% The \emph{abstract} summarizes your paper in a few sentences. What is the problem you want to solve, what is the method you are employing, what are you doing to assess your work, and what is the final outcome.

% The \emph{introduction} should describe the problem that you are solving and why it is important. A good guideline to write a good introduction are the Heilmeier questions:

\emph{摘要}以几句话总结了你的论文。你想解决什么问题,你正在雇用什么方法,你在做什么来评估你的工作,最后的结果是什么。

\emph{前言}应该描述你正在解决的问题,以及为什么它很重要。写一个很好的介绍的好方法是Heilmeier的问题:

\begin{enumerate}
% \item What are you trying to do? Articulate your objectives using absolutely no jargon.
% \item How is it done today, and what are the limits of current practice?
% \item What's new in your approach and why do you think it will be successful?
% \item Who cares?
% \item If you're successful, what difference will it make?
% \item What are the midterm and final ``exams'' to check for success?

\item 你想做什么? 用绝对没有行话说明你的目标。
\item 今天怎么样,现在的做法有什么限制?
\item 你的方法有什么新鲜事,你为什么认为它会成功?
\item 谁在乎?
\item 如果你成功,它会有什么不同?
\item 什么是中期和最后的“考试”来检查成功?
\end{enumerate}

% Originally conceived for proposal writing by the head of DARPA, there are additional questions including ``What will it cost?'', ``How long will it take?'', and ``What are the risks and pay-off'', which are left out for the purpose of writing a research paper. In the context of scientific research, the question ``What are you trying to do?'' is best answered in the form of a \emph{hypothesis}, see below. 

% The \emph{materials \& matters} section describes all the tools that you used to solve your problem, as well as your original contribution, e.g., an algorithm that you came up with. This section is hardly ever labeled as such, but might consist of a series of individual section describing the robotic platform you are using, the software packages, and flowcharts and descriptions on how your system works. Make sure you motivate your design choices using conclusive language or experimental data. Validating these design choices could be your first results. 

% The \emph{results} section contains data or proofs on how to solve the problem you addressed or why it cannot be solved. It is important that your data is conclusive! You have to address concerns that your results are just a lucky coincidence. You therefore need to run multiple experiments and/or formally prove the workings of your system either using language or math, see also Section \ref{sec:stattest}.

最初是由DARPA负责人提出的提案撰写,还有其他问题,包括“它会花费多少?”,“需要多长时间?”和“什么是风险和回报”,为了撰写研究论文而被排除在外。在科学研究的背景下,问题“你想做什么?”最好以\emph{假设}的形式回答,见下文。

\emph{materials \& matters}部分描述了您用来解决问题的所有工具,以及您的原始贡献,例如您提出的算法。本节几乎没有被标记为这样,但可能包含一系列描述您正在使用的机器人平台的单独部分,软件包以及系统如何工作的流程图和描述。确保您使用确凿语言或实验数据激励您的设计选择。验证这些设计选择可能是您的第一个结果。

\emph{results}部分包含有关如何解决您解决的问题或为什么无法解决的数据或证明。重要的是你的数据是决定性的!你必须解决你的结果只是一个幸运的巧合。因此,您需要运行多个实验和/或正式使用语言或数学证明您的系统的运作,另请参阅\ref{sec:stattest}部分。

% The \emph{discussion} should address limitations of your approach, the conclusiveness of its results, and general concerns someone who reads your work might have. Put yourself in the role of an external reviewer who seeks to criticize your work. How could you have sabotaged your own experiment? What are the real hurdles that you still need to overcome for your solution to work in practice? Criticizing your own  work does not weaken it, it makes it stronger! Not only does it become clear where its limitations are, it is also more clear where other people can step in. 

% The \emph{conclusion} should summarize the contribution of your paper. It is a good place to outline potential future work for you and others to do. This future work should not be random stuff that you could possibly think about, but come out of your discussion and the remaining challenges that you describe there. Another way to think about is that the ``future work'' section of your conclusion summarizes your discussion.

% It is important not to mix the different sections up. For example, your result section should exclusively focus on describing your observations and reporting on data, i.e., facts. Don't conjecture here why things came out as they are. You do this either in your hypothesis --- the whole reason you conduct experiments in the first place --- or in the discussion. Similarly, don't provide additional results in your discussion section.

% Try to make the paper as accessible to as many reader styles and attention spans as possible. While this sounds impossible at first, a good way to address this is to think about multiple avenues a reader might take. For example, the reader should get a pretty comprehensive picture on what you do by just reading the abstract, just reading the introduction, or just reading all the figure captions. (Think about other avenues, every one you address makes your paper stronger.) It is often possible to provide this experience by adding short sentences  that quickly recall the main hypothesis of your work. For example, when describing your robotic platform in the materials section, it does not hurt to introduce the section by something like ``In order to show that [the main hypothesis of our work], we selected...''. Similarly, you can try to read through your figure captions if they provide enough information to follow the paper and understand its main results on their own. Its not a problem to be repetitive in a scientific paper, stressing your one-sentence elevator pitch (or hypothesis, see below) throughout the paper is actually a good thing.

\emph{讨论}应该解决您的方法的局限性,其结果的结论以及阅读您的工作的人的一般关切。把自己置于外部评审员的角色,他试图批评你的工作。你如何破坏自己的实验?您仍然需要克服的解决方案在实际工作中真正的障碍是什么?批评你自己的工作不会削弱它,它使它更强大!不仅如此,它的局限性越来越明确,其他人也可以加入其中。

\emph{结论}应该总结你的论文的贡献。这是一个很好的地方,为您和其他人做好未来的潜在工作。这个未来的工作不应该是你可能想到的随机的东西,而是出于你的讨论和你在那里描述的剩下的挑战。另一种想法是,您的结论的“未来工作”部分将总结您的讨论。

重要的是不要混合不同的部分。例如,您的结果部分应专注于描述您的观察和报告数据,即事实。不要在这里猜想为什么事情出来了。你可以在你的假设---你首先进行实验的全部原因---或讨论中做到这一点。同样,不要在您的讨论部分提供其他结果。

尽可能使纸张尽可能多地读取读者风格和注意力范围。虽然这听起来是不可能的,但是解决这个问题的好办法是考虑读者可能会遇到的多种途径。例如,读者应该通过阅读摘要,阅读简介或只阅读所有图形字幕,获得关于您所做的内容的全面了解。(想想其他途径,你所处理的每一个都会使你的论文更加强大。)通常可以通过添加简短的句子来提供这种经验,快速回顾你的工作的主要假设。例如,当在材料部分描述你的机器人平台时,不要像“为了表明我们工作的主要假设”那样介绍这个部分,我们选择了...“。同样,如果您提供足够的信息来跟踪论文并了解其主要结果,您可以尝试阅读您的图形字幕。在科学论文中不重要的问题是,在整篇论文中强调你的一句话电梯音调(或假设,见下文)实际上是一件好事。

% \section{Hypothesis: Or, what do we learn from this work?}
% Classically, a hypothesis is a proposed explanation for an observed phenomenon. From this, the hypothesis has emerged as the corner stone of the scientific method and is a very efficient way to organize your thoughts and come up with a one sentence summary of your work. A proper formulation of your hypothesis should directly lead to the method that you have chosen to test your hypothesis. A good way to think about your hypothesis is ``What do you want to learn?'' or ``What do we learn from this work?''.

% It can be somewhat hard to actually frame your work into a single sentence, so what to do if a single hypothesis seems not to apply? One reason might be that you are actually trying to accomplish too many things. Can you really describe them all in depth in a 6-page document? If yes, maybe some are very minor compared to the others.  If this is the case, they are either supportive of your main idea and can be rolled into this bigger piece of work or they are totally disconnected. If they are disconnected, leave them out for the sake of improving the conciseness of your main message. Finally, you might feel that you don't have a main message, but consider all the things you done equally worthy, and despite answering the Heilmeier questions you cannot fill up more than three pages. In this case you might consider picking one of your approaches and dig deeper by comparing it with different methods.

% Being able to come up with a one-sentence elevator pitch framed as a hypothesis will actually help you to set the scope of the work that you need to do for a research or class project. How good do you need to implement, design or describe a certain component of your project? Well, good enough to follow through with your research objective.

\section{假设:或者,我们从这项工作中学到什么?}
在经典上,假设是对观察现象的提出的解释。从这一点出发,这个假设已经成为科学方法的基石,是组织思想的一个非常有效的方法,并提出一个句子的总结。您的假设的适当公式应直接导致您选择测试您的假设的方法。想想你的假设的一个好方法是“你想学什么?”或者“我们从这项工作中学到什么?”。

实际上将你的工作整理成一个句子可能会有点困难,所以如果一个假设似乎不适用,怎么办?一个原因可能是你实际上想要完成太多的事情。你真的可以在6页的文档中深入描述这些文件吗?如果是的话,也许有些与其他人相比很小。如果是这种情况,他们或者支持你的主要思想,并且可以卷入这个更大的工作,或者完全断开连接。如果它们断开连接,请留出来,以提高主要信息的简洁性。最后,你可能觉得你没有一个主要的信息,但考虑所有你做的同样值得的,尽管回答了Heilmeier的问题,你不能填写三个以上的页面。在这种情况下,您可以考虑选择一种方法,并通过与不同的方法进行比较来深入挖掘。

能够提出一个单句话的电梯音调作为一个假设,实际上可以帮助你设定一个研究或课堂项目所需的工作范围。您需要实现,设计或描述项目的某个组件有多好?好的,跟你的研究目标一致。

% \section{Survey and Tutorial}
% The goal of a \emph{survey} is to provide an overview over a body of work --- potentially from different communities --- and classify it into different categories. Doing this synthesis and establishing common language and formalism is the survey's main contribution.  A survey following such an outline is a possible deliverable for an independent study or a PhD prelim, but it does not lend itself to describe your efforts on a focused research project. Rather, it might result from your involvement in a relatively new area in which you feel important connections between disjoint communities and common language have not been established. 

% A different category of survey critically examines concurring methods to solve a particular problem. For example, you might have set out to study manipulation, but got stuck in selecting the right sensor suite from the many available options. What sensor is actually best to accomplish a specific task? A survey which answers this question experimentally will follow the same structure as a research paper (see above).

% A \emph{tutorial} is closely related to a survey, but focuses more on explaining specific technical content, e.g, the workings of a specific class of algorithms or tool, commonly used in a community. A tutorial might be an appropriate way to describe your efforts in a research project, which can serve as illustration to explain the workings of a specific method you used.

\section{调查和教程}
\emph{survey}的目标是概述一系列可能来自不同社区的工作-并将其分类为不同的类别。调查的主要贡献是综合和建立共同的语言和形式主义。按照这样的大纲进行的调查是独立研究或PhD初步的可能交付成果,但它不适合描述您在重点研究项目上的努力。相反,这可能是由于您参与了一个相对较新的领域,您认为不相交社区和共同语言之间的重要联系尚未建立。

不同类别的调查对于解决特定问题进行了批判性的考察。例如,您可能已经开始研究操纵,但是从许多可用选项中选择了正确的传感器套件。什么传感器实际上是最好的完成一个特定的任务?实验回答这个问题的调查将遵循与研究论文相同的结构(见上文)。

一个\emph{tutorial}与一项调查密切相关,但更多地侧重于解释特定的技术内容,例如社区中通常使用的特定类别的算法或工具的工作。教程可能是描述您在研究项目中所做努力的适当方式,可以作为说明您使用的特定方法的工作原理的说明。

% \section{Writing it up!}
% Writing a research report that contains equations, figures and references requires some tedious book-keeping. Although technically possible, word processing programs quickly reach their limitations and will lead to frustration. In the scientific community \LaTeX~ has emerged as a quasi standard for typesetting research documentation. \LaTeX~ is a mark-up language that strictly divides function and layout. Rather than formatting individual items as bold, italic and the like, you mark them up as emphasized, section head etc, and specify how things look elsewhere. This is usually provided by a template provided by the publisher (or your own). While \LaTeX~ has quite a learning curve compared to other word processing software, it is quickly worth the effort as soon as you need to start worrying about references, figures or even indices. 

\section{写出来!}
撰写包含方程式,数字和参考文献的研究报告需要一些繁琐的书面保存。虽然技术上可能,文字处理程序很快达到其限制,并将导致沮丧。在科学界,\LaTeX~ 已经成为排版研究文献的准标准。\LaTeX~是严格划分功能和布局的标记语言。您不必将单个项目格式化为粗体,斜体等,而是将其标记为强调部分头等,并指定其他位置。这通常由发布商(或您自己的)提供的模板提供。虽然\LaTeX~与其他文字处理软件相比具有相当的学习曲线,但一旦您需要开始担心参考文献,数字甚至索引,就很快就值得一试。

\section*{进一步阅读}

\begin{itemize}
\item W. Strunk and E. White. The Elements of Style (4th Edition). Longan, 1999.
\item T. Oetiker, H. Partl, I. Hyna and E. Schlegl. The Not So Short Introduction to \LaTeXe. Available online.
\end{itemize}
