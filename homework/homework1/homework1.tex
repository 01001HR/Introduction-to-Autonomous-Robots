%% Based on a TeXnicCenter-Template by Tino Weinkauf.
%%%%%%%%%%%%%%%%%%%%%%%%%%%%%%%%%%%%%%%%%%%%%%%%%%%%%%%%%%%%%

%%%%%%%%%%%%%%%%%%%%%%%%%%%%%%%%%%%%%%%%%%%%%%%%%%%%%%%%%%%%%
%% HEADER
%%%%%%%%%%%%%%%%%%%%%%%%%%%%%%%%%%%%%%%%%%%%%%%%%%%%%%%%%%%%%
\documentclass[letter,twoside,11pt]{article}
% Alternative Options:
%	Paper Size: a4paper / a5paper / b5paper / letterpaper / legalpaper / executivepaper
% Duplex: oneside / twoside
% Base Font Size: 10pt / 11pt / 12pt


%% Language %%%%%%%%%%%%%%%%%%%%%%%%%%%%%%%%%%%%%%%%%%%%%%%%%
\usepackage[USenglish]{babel} %francais, polish, spanish, ...
\usepackage[T1]{fontenc}
\usepackage[ansinew]{inputenc}

\usepackage{lmodern} %Type1-font for non-english texts and characters


%% Packages for Graphics & Figures %%%%%%%%%%%%%%%%%%%%%%%%%%
\usepackage{graphicx} %%For loading graphic files
%\usepackage{subfig} %%Subfigures inside a figure
%\usepackage{pst-all} %%PSTricks - not useable with pdfLaTeX

%% Please note:
%% Images can be included using \includegraphics{Dateiname}
%% resp. using the dialog in the Insert menu.
%% 
%% The mode "LaTeX => PDF" allows the following formats:
%%   .jpg  .png  .pdf  .mps
%% 
%% The modes "LaTeX => DVI", "LaTeX => PS" und "LaTeX => PS => PDF"
%% allow the following formats:
%%   .eps  .ps  .bmp  .pict  .pntg


%% Math Packages %%%%%%%%%%%%%%%%%%%%%%%%%%%%%%%%%%%%%%%%%%%%
\usepackage{amsmath}
\usepackage{amsthm}
\usepackage{amsfonts}


%% Line Spacing %%%%%%%%%%%%%%%%%%%%%%%%%%%%%%%%%%%%%%%%%%%%%
%\usepackage{setspace}
%\singlespacing        %% 1-spacing (default)
%\onehalfspacing       %% 1,5-spacing
%\doublespacing        %% 2-spacing


%% Other Packages %%%%%%%%%%%%%%%%%%%%%%%%%%%%%%%%%%%%%%%%%%%
%\usepackage{a4wide} %%Smaller margins = more text per page.
%\usepackage{fancyhdr} %%Fancy headings
%\usepackage{longtable} %%For tables, that exceed one page


%%%%%%%%%%%%%%%%%%%%%%%%%%%%%%%%%%%%%%%%%%%%%%%%%%%%%%%%%%%%%
%% Remarks
%%%%%%%%%%%%%%%%%%%%%%%%%%%%%%%%%%%%%%%%%%%%%%%%%%%%%%%%%%%%%
%
% TODO:
% 1. Edit the used packages and their options (see above).
% 2. If you want, add a BibTeX-File to the project
%    (e.g., 'literature.bib').
% 3. Happy TeXing!
%
%%%%%%%%%%%%%%%%%%%%%%%%%%%%%%%%%%%%%%%%%%%%%%%%%%%%%%%%%%%%%

%%%%%%%%%%%%%%%%%%%%%%%%%%%%%%%%%%%%%%%%%%%%%%%%%%%%%%%%%%%%%
%% Options / Modifications
%%%%%%%%%%%%%%%%%%%%%%%%%%%%%%%%%%%%%%%%%%%%%%%%%%%%%%%%%%%%%

%\input{options} %You need a file 'options.tex' for this
%% ==> TeXnicCenter supplies some possible option files
%% ==> with its templates (File | New from Template...).



%%%%%%%%%%%%%%%%%%%%%%%%%%%%%%%%%%%%%%%%%%%%%%%%%%%%%%%%%%%%%
%% DOCUMENT
%%%%%%%%%%%%%%%%%%%%%%%%%%%%%%%%%%%%%%%%%%%%%%%%%%%%%%%%%%%%%
\begin{document}

\title{Homework 1}
\author{Introduction to Robotics}
%\date{} %%If commented, the current date is used.
\maketitle

\begin{enumerate}
\item What are the degrees of freedom of a standard, hand-pushed lawnmower? Why are you still able to mow your entire lawn?
\item What are the maximum degrees of freedom for objects driving on the plane?
\item 
\begin{enumerate}
 \item Write out the entries of a rotation matrix $^A_BR$ assuming basis vectors $X_A$, $Y_A$, $Z_A$, and $X_B$, $Y_B$, $Z_B$. 
 \item Express $\hat{X}_B=[0, 1, 0]^T$ in frame $\{A\}$.
 \item Write out the entries of rotation matrix $^B_AR$.
 \item Verify that your solution for $^A\hat{X}_B=^A_BR[0, 1, 0]^T$ indeed reduces to $[0, 1, 0]^T$ when re-expressed in coordinate system $\{B\}$. Tip: Express multiplications as scalar products and remember that the scalar product of orthogonal vectors is 0 and that of parallel vectors is 1.
\end{enumerate} 
\item Consider a tri-cycle with two independent standard wheels in the rear and the stearable front-wheel. Chose a suitable coordinate system and use $\phi$ as the steering wheel angle and wheel-speed $\dot{\omega}$ (only the steared front-wheel is driven). Provide forward and inverse kinematics. 

\end{enumerate}

%% Title Page %%%%%%%%%%%%%%%%%%%%%%%%%%%%%%%%%%%%%%%%%%%%%%%
%% ==> Write your text here or include other files.

%% The simple version:


%% The nice version:
%\input{titlepage} %%You need a file 'titlepage.tex' for this.
%% ==> TeXnicCenter supplies a possible titlepage file
%% ==> with its templates (File | New from Template...).
\end{document}

